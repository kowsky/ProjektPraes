\documentclass{article}

\usepackage[utf8]{inputenc}
\usepackage{graphicx}
\usepackage[german]{babel}
\usepackage{algpseudocode}
\usepackage{algorithm}
\usepackage{amsmath}
\usepackage{hyperref}
\usepackage{listings}

%\usepackage{algorithmic}
\usepackage{cite}

\setlength{\parindent}{0pt}

\title{Proteinsequenz-Vergleich mit \emph{KSEARCH} und \emph{SANS}}
\author{Meike Bruns, Florian Markowsky, Michael Spohn}

\begin{document}

\maketitle
\thispagestyle{empty}
\begin{abstract}
In unserer Projektarbeit beschäftigten wir uns mit der Implementierung zweier Algorithmen zur Ähnlichkeitsbewertung von Proteinsequenzen, \emph{SANS} und \emph{KSEARCH}, innerhalb der GenomeTools Umgebung. Die von J. Patrik Koskinen und Liisa Holm entwickelten Algorithmen weisen Proteinpaaren aufgrund von gemeinsam enthaltenen $k$-meren (\emph{KSEARCH}) bzw. von Nachbarschaften der Proteinsuffixe im Suffixarray (\emph{SANS}) Ähnlichkeitsscores zu. Koskinen und Holm erreichten so bei kurzen Laufzeiten eine mit BLAST vergleichbare Sensitivität bei Proteinen mit 50-100\% Sequenzidentität. Unsere Implementierung der Algorithmen liefert mit der Originalimplementierung vergleichbare Ergebnisse, die von Koskinen und Holm erreichten Laufzeiten konnten jedoch nicht reproduziert werden.
\end{abstract}
\newpage

\tableofcontents
\thispagestyle{empty}
\newpage

\section{Einleitung: Sequenzvergleiche}

Eine bei der Sequenzanalyse am häufigsten anfallenden Aufgaben ist der Vergleich
von Sequenzen. 
Sowohl bei DNA als auch bei Proteinsequenzen geben Sequenzvergleiche Aufschluss über die Ähnlichkeiten von Sequenzen, die dazu herangezogen werden, Verwandschaftsbeziehungen zwischen den einzelnen Sequenzen oder konservierte Regionen zu ermitteln. Bei unbekannten Proteinsequenzen lassen sich so beispielsweise Vorhersagen zur Funktion oder zur Struktur machen. Da exakte Sequenzvergleiche sehr zeitaufwendig sind, wurden verschiedene weniger exakte, dafür schnellere Algorithmen entwickelt.\\
Viele dieser Algorithmen basieren auf Sequenzalignments, wie der FASTA-Algorithmus \cite{FASTA} oder BLAST \cite{BLAST}, der sich als Standard für den Proteinvergleich etabliert hat.
Andere Algorithmen, wie z.B. USEARCH \cite{USEARCH} berechnen Ähnlichkeiten von Sequenzen dagegen unabhängig von Alignments anhand der Ähnlichkeiten von Feature-Vektoren. 
Feature-Vektoren einer Sequenz können z.B. die in ihr enthaltenen k-mere oder Substrings sein. 
Um die Ähnlichkeit zweier Sequenzen zu bestimmen, werden ihre Feature-Vektoren verglichen und daraus ein Ähnlichkeits-Score ermittelt.\\
Während alignmentbasierte Methoden für viele Zwecke eingesetzt werden können, sind Feature-Vektor basierte Scores hauptsächlich in Anwendungsbereichen wie der nearest-neighbor-Analyse vorteilhaft. 
Dadurch, dass sie ganz ohne Alignment auskommen, sind sie meist sehr schnell und lassen sich daher auch für große Datenmengen gut verwenden. 
Dass die berechneten Scores keine detaillierten Informationen darüber enthalten,
wo zwei Sequenzen gleich und wo verschieden sind, ist in nearest-neighbor-Analysen wie z.B. dem Clustering kein Nachteil. 
Hier ist vor allem von Interesse, dass hohe Scores ein verlässlicher Indikator für eine hohe Ähnlichkeit sind.\\
Im folgenden stellen wir unsere Implementierung von zwei von J. Patrik Koskinen und Liisa Holm entwickelten Algorithmen - \emph{SANS} (Suffix Array Neighborhood Search) und \emph{KSEARCH} - vor, die Feature-Vektor basierte Scores für Protein-Sequenzpaare ermitteln. Koskinen und Holm erreichten mit \emph{SANS} für Proteinsequenzen, deren Sequenzidentität zwischen 50 und 100\% liegt, eine mit Blast vergleichbare Sensitivität. In Abhängigkeit vom gewählten $k$-Wert konnten mit \emph{KSEARCH} ähnliche Sensitivitätswerte erreicht werden. Dieses Ergebnis ist für die Vergleiche von Proteinsequenzen überraschend, denn wegen der hohen Mutationswahrscheinlichkeiten einzelner Aminosäuren werden in BLAST zur Bestimmung von Ähnlichkeiten zwischen Proteinen Substitutionsmatrizen verwendet, die die Mutationswahrscheinlichkeiten der Aminosäuren in die Ähnlich\-keitsberechnung einbeziehen. 
Diese Anpassung ist bei den vorgestellten Algorithmen nicht möglich.


\section{Algorithmen \& Implementierung}

\subsection{KSEARCH}
\label{ksearch}

Der in \cite{Holm} vorgestellte \emph{KSEARCH}-Algorithmus ermittelt Ähnlichkeitsscores für Proteinpaare anhand der $k$-mere, die in den Sequenzen beider Proteine vorkommen. Für jedes mögliche $k$-mer der Länge $k$ wird die Häufigkeit seines Auftretens im query-Protein mit der Häufigkeit seines Auftretens im database-Protein multipliziert. Die Summe dieser Produkte über alle $k$-mere ergibt den \emph{KSEARCH}-Score:

\begin{equation*}
\text{\emph{KSEARCH(query,database)}} = \sum_{w \in \mathcal A^k} G_k(\text{\emph{query}})(w) \cdot G_k(\text{\emph{database}})(w)
\end{equation*}  
mit
\emph{ $G_k(query)(w)$} = Häufigkeit des Vorkommens des $k$-mers $w$ im query-Protein\\ \\

Um durch kurze Tandem Repeats verursachte, künstlich erhöhte Scores zu vermeiden, werden, wie in \cite{Holm}, $k$-mere, die an 1., 2., und 3., bzw. an 1., 3., und 5. Position dieselben Zeichen enthalten, nicht bei der Berechnung des \emph{KSEARCH}-Scores berücksichtigt.\\
Koskinen und Holm benutzen zu Berechnung der \emph{KSEARCH} - Scores Suffixarrays, und erreichen so eine theoretische Laufzeitkomplexität von $O(|q_1..q_n| \cdot |d_1..d_n|)$, also in der Gesamtlänge der query- und database-Proteine. Laut Koskinen und Holm liegt der Vorteil der Implementierung mit Suffixarrays darin, dass der Parameter für $k$ frei gewählt werden kann. \\
Mit der q-gram-Distanz existiert bereits ein Modell, dass auf der Idee basiert,
Sequenzpaaren anhand vom gemeinsamen $k$-meren (bzw. $q$-Worten) Änhnlichkeits\-scores zuzuweisen. 
Im Gegensatz zum \emph{KSEARCH} Score ergibt sich der Score bei der q-gram-Distanz aus der Summe der Differenzbeträge $|G_q(query)(w) - G_q(database)(w)|$ der Häufigkeiten der $k$-mere im query- und database-Protein. 
Bei der q-gram-Distanz konnte durch die Verwendung von $q$-Wort Profilen eine
Laufzeit von $O(|n|+|m|+\mathcal{A}^q)$ erreicht werden, wobei $|m|$ und $|n|$
die Längen der Proteine, $\mathcal{A}^q$ die Anzahl der möglichen q-Worte
darstellen \cite{GSA}. Es erschien uns daher sinnvoll, q-Wort-Profile auch für die Berechnung von \emph{KSEARCH}-Scores zu nutzen.\\
Für das Erstellen eines $k$-mer- bzw. q-Wort-Profils wird jedes mögliche $k$-mer durch einen Integerwert codiert. Ein Array von der Größe $\mathcal{A}^k$ enthält an jeder Position $i$ die Anzahl des dazugehörigen $k$-mers mit dem Integercode $i$, und wird daher mit Null initialisiert. Durch eine geschickte Codierung können die Codes der einzelnen, in der Sequenz enthaltenen $k$-mere in konstanter Zeit ermittelt werden, so dass die Berechnung der $k$-mer-Profile in ${O(|q_1..q_n|)}$ erfolgt. Da der \emph{KSEARCH}-Score zur Bewertung von Proteinsequenzpaaren genutzt werden soll, ergibt sich bei der Verwendung von Arrays zur Speicherung der $k$-mer Profile der erforderliche Speicherplatz von $\mathcal{A}^k = 20^k$ pro Profil. Darüber hinaus sind die Proteinsequenzen relativ kurz, so dass die berechneten $k$-mer Profile nur spärlich besetzt sind. Daher nutzen wir Hashes, um die $k$-mer-Profile zu speichern. \\
Unsere Implementierung des \emph{KSEARCH}-Algorithmus erhält die durch das GenomeTool encseq encode konkatenierten und komprimierten query- und database-Proteinsequenzen als Eingabe. Mit dem GenomeTools Modul GtKmercodeiterator werden zunächst die $k$-mere jedes query-Proteins extrahiert und mit der GenomeTools Funktion GtHashmap in einem Hash gespeichert.
Dann wird für jedes database-Protein das $k$-mer-Profil als Hash angelegt und der \emph{KSEARCH}-Score mit allen query-Poteinen ermittelt und ausgegeben. Da das $k$-mer-Profil des database-Proteins nicht noch einmal gebraucht wird, braucht es nicht gespeichert zu werden. Das Berechnen des $k$-mer-Profils eines database-Proteins $d$ geschieht in $O(|d|)$, für alle database-Proteine also in $O(d_1..d_n)$. Dazu kommt die Berechnung der \emph{KSEARCH}-Scores, die, da das $k$-mer Profil als Hash vorliegt, für ein einzelnes Proteinpaar linear in der Anzahl unterschiedlicher $k$-mere im database-Protein $O(kVariety(d))$ ist, für alle Proteinpaare also in $O(|D| \cdot |Q| \cdot kVariety(d))$ liegt. Insgesamt erhalten wir also eine Laufzeitkomplexität von $O(|q_1..q_n|+|d_1..d_n|+|D| \cdot |Q| \cdot kVariety(d))$.



\begin{algorithm}
  \caption{Pseudocode unserer \emph{KSEARCH}-Implementierung}
\begin{algorithmic}
  \Function{KSEARCH}{query-proteins $Q$, database-proteins $D$}
    \For{each query-protein $q$}  
      \State calculate kmer-profile \Comment $O(|q|)$
      \State store kmer-profile in hashtable
    \EndFor \Comment $\mathbf{O(|q_1..q_n|)}$
    \For{each database-protein $d$} 
      \State calculate kmer-profile \Comment $O(|d|)$
      \For{i in $Q$}
        \State calculate $KSEARCH(d,q_{i})$ \Comment $O(kVariety(d))$   
        \State output $KSEARCH(d,q)$
      \EndFor \Comment $\mathbf{O(|Q| \cdot kVariety(d))}$
    \EndFor   \Comment $\mathbf{O(|d_1..d_n|+|D| \cdot |Q| \cdot kVariety(d))}$  
  \EndFunction \Comment overall time complexity: $\mathbf{O(|q_1..q_n|+|d_1..d_n|+|D| \cdot |Q| \cdot kVariety(d))}$
\end{algorithmic}
\end{algorithm}

\subsection{SANS}
\label{sans}

Der \emph{SANS}-Algorithmus bewertet Proteinpaare, deren Suffixe längste gemeinsame
Präfixe haben, mit einem positiven Score. Suffixarrays haben die Eigenschaft,
dass Suffixe, die im Suffixarray nebeneinanderliegen, typischerweise lange
gemeinsame Präfixe haben. Für die  \emph{SANS}-Scores wird diese Eigenschaft dahingehend
genutzt, dass Proteinpaare von query- und database-Proteinen, deren Suffixe im
Suffixarray nahe beieinander liegen, mit einem hohen Score bewertet werden.\\
Zur Berechnung der  \emph{SANS}-Scores werden zunächst alle query-Proteine $q_1..q_n$ konkateniert, ebenso alle database-Proteine $d_1..d_n$. 
Aus den konkatenierten Sequenzen werden die Suffixarrays \emph{Suffixarray$_Q$} und \emph{Suffixarray$_D$} erstellt. 
Nun wird für jeden Suffix \emph{suffix$_{qx}$} im \emph{Suffixarray$_Q$} die
Position $i$ im \emph{Suffixarray$_D$} ermittelt, an der der Suffix gemäß der lexikographischen Ordnung einzusortieren wäre. 
Ein Fenster der Größe $2h$ wird so über diese Position gelegt, dass es die $h$
Positionen vor und die $h$ Positionen nach dem hypothetisch einsortierten
\emph{suffix$_{qx}$} überdeckt. 
Anhand dieses Fensters werden die \emph{SANS}-Scores aktualisiert: 
Jeder Suffix \emph{suffix$_{dy}$} eines Proteins $y$ im Fenster erhöht den  \emph{SANS}-Score-Zähler der dazugezugehörigen Proteinkombination $(query_x,database_y)$ um eins. \\
Wenn für alle Suffixe aus \emph{Suffixarray$_Q$} die Scores aktualisiert wurden, kann der  \emph{SANS}-Score ausgegeben werden.

Formal berechnet sich der \emph{SANS(query,database)} also folgendermaßen:\\

\begin{equation*}
  \begin{split}
\text{\emph{SANS(query,database)}} = \\ \sum_{s \in \text{\emph{suffixe$_q$}}} \sum_{j=-h}^{+h} \text{\emph{id(database,Suffixarray$_D[$position$_D(s)  +j]$.protein)}} 
  \end{split}
\end{equation*}

mit\\
\emph{suffixe\_q} - alle Suffixe aus $q$,\\
\emph{position$_D$(s)} - theoretische Position des Suffixes $s$ in \emph{Suffixarray$_D$} \\
\emph{Suffixarray$_D[x]$.protein} - database-Protein, zu dem der Suffix an der Position $x$ des \emph{Suffixarray$_D$} gehört. \\

und der Identitätsfunktion\\

\begin{equation*}
id(a,b)=\begin{cases}
  1,  & if~a=b\\
  0,  & if~a\ne b\\
\end{cases}
\end{equation*}

Auch bei der Berechnung des \emph{SANS}-Scores werden Suffixe, die an 1., 2., und 3., bzw. an 1., 3., und 5. Position dieselben Zeichen enthalten, nicht bei der Berechnung des Scores berücksichtigt, um eine künstliche Erhöhung des Scores durch kurze Tandem Repeats zu vermeiden.\\
Koskinen und Holm setzen die Fenstergröße $h$ standardmäßig auf die Größe der als
Ausgabe gewünschten Anzahl von Proteinpaaren pro query Protein. 
Das Ziel dabei ist es, jedem der ausgegebenen $h$ Proteine, die Möglichkeit zu
geben, im Fenster aufzutauchen, selbst wenn die Suffixe \emph{suffix$_{d}$}
aller $h$ Proteine beispielsweise vor der Position $i$ des Suffixes
\emph{suffix$_{qx}$} auftauchen. 
Da wir mehrere Ausgabevarianten implementieren, haben wir uns dagegen entschieden, den Standardwert für $h$ an die Ausgabe zu koppeln.\\
Zum Erstellen der Suffixarrays aus den query- und database-Proteinen nutzen wir
den in den GenomeTools enthaltenen Suffixerator, dessen Laufzeitkomplexität
linear in der Länge der konkatenierten Proteinsequenzen $O(|d_1..d_n|+|q_1..q_n|)$ ist.
Die Suffixarrays stellen die Eingabe für unseren  \emph{SANS}-Algorithmus dar.
Da die endgültigen Scores erst feststehen, wenn alle Suffixe im
\emph{Suffixarray$_Q$}
verarbeitet wurden, muss zur Ermittelung der Scores eine Scorematrix der Größe
$|Q|\cdot|D|$ angelegt und mit Null initialisiert werden, wobei $|Q|$ die Anzahl
der query-, $|D|$ die Anzahl der database-Proteine beschreibt. Dies benötigt
$O(|Q|\cdot|D|)$ Zeit. Jetzt ermitteln wir die Positionen der Suffixe
i\emph{Suffix$_{qx}$} in \emph{Suffixarray$_D$}. Dazu laufen wir einmal durch
den gesamten \emph{Suffixarray$_Q$}. Da die Suffixe in den Suffixarrays
lexikographisch geordnet sind, brauchen wir nicht für jeden \emph{Suffix$_{qx}$}
durch den gesamten \emph{Suffixarray$_D$} zu laufen, sondern können mit der
Suche nach der Position von \emph{Suffix$_{qx}$} an der Position beginnen, an
der wir auch \emph{Suffix$_{qx-1}$} eingeordnet haben. Diese Position merken wir
uns in der Variablen \emph{position}. Damit laufen wir einmal durch den gesamten
\emph{Suffixarray$_Q$} sowie insgesamt genau einmal durch den gesamten
\emph{Suffixarray$_D$} und bleiben so in $O(|q_0..q_n| + |d_0..d_n|)$.
Zusätzlich müssen immer, wenn wir eine Position gefunden haben, die Scores $2
\cdot h$ Proteinpaaren in konstanter Zeit aktualisiert werden, so dass sich für
die Berechnung der Scores eine Laufzeit von ${O(|q_0..q_n|\cdot2h +
|d_0..d_n|)}$ ergibt. Zusammen erhalten wir also eine Laufzeitkomplexität von ${O(|q_0..q_n|\cdot2h + |d_0..d_n|+|Q|\cdot|D|)}$


\begin{algorithm}
  \caption{Pseudocode unserer \emph{SANS}-Implementierung}
 \begin{algorithmic}
   \Function{SANS}{$query proteins Q, database proteins D, windowsize h$}
     \State calculate $suffixarray_Q$ for query proteins \Comment $O(|d_1..d_n|)$
     \State calculate $suffixarray_D$ for database proteins \Comment $O(|q_1..q_n|)$
     \State initialize score matrix \Comment $O(|Q|\cdot|D|)$
     \State initialize $position = 0$
     \For{each $suffix_{qx}$ from 0 - $suffixarray_Q$.length}
       \State  $position$ = \Call {findPosition}{$suffix$, $suffixarray_D$, $position$} 
       \State shift $2h$ window over $position$ 
       \For{each $suffix_{dx}$ in window}
         \State update SANS-scores for  $(q_x,d_x)$
       \EndFor \Comment $\mathbf{O(2h)}$
     \EndFor \Comment $\mathbf{O(|q_0..q_n|\cdot2h + |d_0..d_n|)}$
    \EndFunction  \Comment{overall time complexity:$\mathbf{O(|q_0..q_n|\cdot2h + |d_0..d_n|+|Q|\cdot|D|)}$}   \\
    
    \Function{findPosition}{$suffix$, $suffixarray_D$, $position$}
      \For{$i$ in $position$..$suffixarray_D$.length}
        \If{$suffix$ fits in $suffixarrax_D[i]$}
          \State return $i$ 
        \EndIf
      \EndFor 
    \EndFunction
  \end{algorithmic}
\end{algorithm}

\subsection{Software Architektur}

\subsubsection{GenomeTools}

Die GenomeTools sind eine freie Sammlung von Tools, Klassen und Modulen zum
Analysieren von Genomen und zur Entwicklung weiterer Programme zur Genomanalyse \cite{gtools}. Wir implementierten die Algorithmen \emph{SANS} und \emph{KSEARCH} innerhalb der GenomeTools-Umgebung und nutzten dabei folgende der durch GenomeTools bereitgestellten Funktionalitäten:

\paragraph{GtToolbox}
Die GtToolbox ist eine Klasse, die die einzelne Tools enthält. Sie besitzt
Methoden zum parsen der übergebenen Parameter und zum Aufrufen des ausgewählten
Tools. Wir implementierten unsere Algorithmen \emph{SANS} und \emph{KSEARCH} als Tools
innerhalb der GtToolbox \emph{gt\_seqscore}. 

\paragraph{GtEncseq}
GtEncseq encode ist ein Tool, um aus ein oder mehreren Sequenzdateien ein
GtEncseq-Objekt zu erzeugen. Ein Objekt der Klasse GtEncseq besteht aus den
konkatenierten und durch bit-Kompression codierten Sequenzen der Eingabedateien.
Je nachdem, welche Funktionalitäten für ein GtEncseq-Objekt gewünscht werden, werden pro Objekt bis zu vier Dateien angelegt. Mit den verschiedenen Funktionen der GtEncseq-Klasse lassen sich unterschiedliche Eigenschaften der codierten Sequenzen abfragen. Wir verwenden GtEncseq-Objekte als Eingabe für den \emph{KSEARCH}-Algorithmus.

\paragraph{GtKmercodeiterator}
Der GtKmercodeiterator ist eine Klasse, deren Instanzen unter anderem ein GtEncseq-Objekt, eine Startposition und eine Länge für $k$ erhalten. Mit den Klassenfunktionen \emph{gt\_kmercodeiterator\_encseq\_next()} wird das jeweils nächste $k$-mer der GtEncseq als GtKmercode zurückgegeben. Wir benutzen diese Funktionalität, um die $k$-mer-Hashes für den \emph{KSEARCH}-Algorithmus zu erstellen.

\paragraph{GtRankedList und GtRankedListIter}
Die Instanzen der GtRankedList-Klasse sind dynamische Listen fester Größe. In ihnen wird gewährleistet, dass die gespeicherten Objekte in der Liste jeweils die mit dem höchsten Rang sind. Die Größe der Liste sowie die Vergleichsfunktion zur Ermittlung des Ranges sind Klassenvariablen der GtRankedList, zur Eingabe von Objekten in die GtRankedList gibt es eine Klassenfunktion. Um alle in einerGtRankedList gespeicherten Objekte auszugeben, wird die GtRankedList an ein Objekt der Klasse GtRankedListIter übergeben.

\paragraph{GtHashmap}
Die Objekte der GtHashmap-Klasse sind Hashtables. Wir benutzen sie, um die $k$-mer
Profile der Proteinsequenzen zu speichern. Dabei stellt das jeweilige $k$-mer den
Schlüssel und die Anzahl des Auftretens des $k$-mers den zugeordneten Wert dar. Es
werden also nur Informationen über tatsächlich in der Sequenz vorkommende $k$-mere
gespeichert.

\paragraph{GtSuffixerator}
Der GtSuffixerator ist ein Tool, das aus einer Sequenzdatei oder einem GtEncseq-Objekt einen Suffixarray erstellt. Standardmäßig werden für einen Suffixarray fünf Dateien erstellt, zusätzlich können Dateien für die Burrow-Wheeler-Transformation oder den LCP-Table angelegt werden. Wir verwenden den GtSuffixerator, um aus den query- und database-Dateien Suffixarrays als Eingabe für den \emph{SANS}-Algorithmus zu erstellen.

\subsubsection{Struktur}

Wir implementierten die Algorithmen für den \emph{SANS}- und den \emph{KSEARCH}-Score als zwei Tools \emph{gt\_seqscore\_sans.c} und \emph{gt\_seqscore\_ksearch.c} innerhalb der Toolbox \emph{gt\_seqscore} und erhalten so eine logische Trennung. Um Code-Redundanzen zu vermeiden, greifen beide Tools auf eine gemeinsame Codebasis zu, die in \emph{bla\_seqscore.c} implementiert ist. Der Aufruf der Tools erfolgt über die Konsole in der GenomeTools Umgebung und enthält die Toolbox, das Tool, die Namen der zu vergleichenden Encseq-Sequenzen, die Ausgabeoption sowie die Länge der $k$-mere $k$ (\emph{KSEARCH}) bzw. die Fensterweite $h$ (\emph{SANS}) optional kann eine Ausgabedatei angegeben werden.
%TODO Default-Werte?
Für beide Tools gibt es drei Ausgabeoptionen, \emph{-matrix}, \emph{-nbest x} und \emph{-threshold x}. Wenn die Option \emph{-matrix} gewählt wird, werden die Scores aller Proteinpaare ausgegeben, bei der Option \emph{-nbest x} nur die $x$ Paare mit den besten Scores, und bei \emph{-threshold x} alle Paare, deren Score größer oder gleich $x$ ist. \\
Je nachdem, welche Ausgabeoption gewählt wurde, wird in der Runner-Funktion des jeweiligen Tools eine passende Datenstruktur erstellt. Für die \emph{-matrix} Option ist das eine Matrix der Größe $|Q|\cdot|D|$ und für die \emph{-nbest} Option eine GtRankedList. Bei der \emph{-threshold} Option dagegen können die Ergebnisse direkt ausgegeben werden, so dass keine Speicherstruktur angelegt werden muss. Durch den Aufruf eines eintsprechenden Konstruktors aus \emph{bla\_seqscore.c} werden die Strukturen dann gefüllt, so dass für jedes Tool drei verschiedene Konstruktoren existieren:
\emph{gt\_seqscore\_result\_matrix\_new} für die \emph{-matrix} Option, \emph{gt\_seqscore\_ranked\_list\_new} für die \emph{-nBest} Option und \emph{gt\_seqscore\_threshold\_new} für die \emph{-treshold} Option.
Für die Berechnung der Scores wurde für jeden Algorithmus nur jeweils eine Funktion implementiert, \emph{gt\_calculate\_results\_sans} und \emph{gt\_enumerate\_results\_ksearch}. 
Jeder der drei Konstruktoren eines Tools greift für die Berechnung der Scores auf die jeweilige Funktion zu.\\
Da bei der Berechnung der \emph{SANS}-Scores wie in \ref{sans} erläutert die gesamte Score-Matrix bis zum Ende des Algorithmus mitgeführt werden muss, werden in den Konstruktoren \emph{gt\_seqscore\_result\_matrix\_new\_sans()}, \emph{gt\_seqscore\_nbest\_new\_sans()} und  \emph{gt\_seqscore\_threshold\_new\_sans()} zunächst Ergebnismatrizen angelegt und der entsprechende Speicherplatz von $|Q|\cdot|D|$ alloziert. Die Matrizen werden mit einem Aufruf von \emph{calculate\_results\_sans()} gefüllt und anschließend dem Konstruktoraufruf entsprechend aufbereitet. So gibt der \emph{matrix}-Konstruktor die gesamte Matrix zurück, während im \emph{nbest}-Konstruktor aus den Scores in der Matrix eine GtRankedList der $x$ besten Scores und der dazugehörigen Sequenzen erstellt wird, und im \emph{threshold}-Konstruktor alle Sequenzpaare, deren Score über dem gegebenen Schwellwert liegt, direkt ausgegeben werden.\\
Wie in Abschnitt \ref{ksearch} ersichtlich, werden beim \emph{KSEARCH}-Algorithmus die Scores der Sequenzpaare nacheinander berechnet, so dass Speicherplatz gespart werden kann. Nur wenn der Benutzer die \emph{-matrix} Rückgabe gewählt hat, muss also tatsächlich der Speicher für die ganze Ergebnismatrix alloziert werden. Daher werden in den Konstruktoren \emph{gt\_seqscore\_result\_matrix\_new\_ksearch()},\linebreak \emph{gt\_seqscore\_nbest\_new\_ksearch()} und  \emph{gt\_seqscore\_threshold\_new\_ksearch()} ausgabespezifische Ergebnisstrukturen erstellt und mit einer strukturspezifischen Call\-back-Funktion an 
die \emph{enumerate\_results\_ksearch()} Funktion übergeben. Diese berechnet sequentiell die Scores und schreibt sie mittels der Callback-Funktion in die übergebene Struktur. Falls das Tool mit der \emph{-threshold} Option aufgerufen wurde, ist das Übergeben einer Ergebnisstruktur nicht notwendig. Die Ergebnisse werden dann durch die Callback-Funktion direkt ausgegeben. Die Abläufe eines \emph{KSEARCH} Aufrufs sind in den Sequenzdiagrammen \ref{manbes} und \ref{thresh} für die Optionen \emph{-nbest} und \emph{-threshold} verdeutlicht. \\

\begin{center}
  \begin{figure}
    \includegraphics[width = \linewidth]{img/dia2}
    \caption{Struktur der \emph{bla\_seqscore.c}. Für jedes von außen aufrufende Tool sind drei Konstruktoren vorhanden, die jeweils dieselbe Berechnungsfunktion ausführen.}
    \label{seqsc}
  \end{figure}
\end{center}

\begin{center}
  \begin{figure}
    \includegraphics[width = \linewidth]{img/seqscore_seqence_fam.png}
    \caption{Sequenzdiagram der Abfolge der Funktionsaufrufe beim Aufruf von \emph{KSEARCH} mit \emph{-matrix} Option. Die Aufrufe für die \emph{-nbest} Option sind analog.}
    \label{manbes}
  \end{figure}
\end{center}

\begin{center}
  \begin{figure}
    \includegraphics[width = \linewidth]{img/seqscore_sequence_fam_thresh.png}
    \caption{Sequenzdiagram der Abfolge der Funktionsaufrufe beim Aufruf von  \emph{KSEARCH} mit \emph{-threshold} Option. Beim Aufruf der Funktion \emph{enumerate\_results\_ksearch()} wird statt einer Ergebnisstruktur ein NULL-Poiter übergeben, da Scores, die über dem threshold liegen, direkt ausgegeben werden können.}
    \label{thresh}
  \end{figure}
\end{center}



%\begin{center}
%  \begin{figure}
%    \includegraphics[width = \linewidth]{img/dia3}
%    \caption{Aufruf-Reiehenfolge bei Aufruf des \emph{result\_matrix\_new\_ksearch} Konstruktor. Der Konstruktor übergibt der Funktion zur Score-Berechnung
%      die Callback-Funktion \emph{callback\_matrix()} sowie die \emph{ScoreMatrix} als Ergebnisstruktur. Erst in der Callback-Funktion, die in der 
%      Berechnungsfunktion aufgerufen wird, wird die Ergebnisstruktur wieder zu einer Matrix gecastet und befüllt.}
%    \label{kscallback}
%  \end{figure}
%\end{center}



\section{Auswertung und Diskussion}

Um die erstellten Implementierungen von \emph{KSEARCH} und \emph{SANS} zu testen, wurden einerseits Laufzeitvergleiche mit den bereits existierenden Algorithmen aus \cite{Holm}, sowie Trefferquoten-Vergleiche mit bewährten Algorithmen wie SSEARCH und BLAST durchgeführt. Als Query-Datensatz dienten alle 1083 \textit{Rattus rattus}-Proteine aus der NCBI-Datenbank. Als Datenbank wurde die zur Uniprot gehörende Swissprot gewählt, welche ca. 500 000 Proteine enthält.

\subsection{Laufzeiten}

Die im Vergleich zu SSEARCH und BLAST sehr guten Laufzeiten des SANS-Algorithmus aus \cite{Holm}
konnten in unserer Implementierung nicht erreicht werden (Tab. \ref{runtimes}), sind jedoch für den praktischen Einsatz kurz genug.
Bei \emph{KSEARCH} ist dies nicht der Fall. Ist die Laufzeit hier bei kleinen Test-Datensätzen noch vertretbar, steigt sie auf unserem realitätsnahen
Testdatensatz von 1083 Rattus-Proteinen mit der Swissprot Datenbank auf über 500 Minuten, was unsere \emph{KSEARCH} Implementierung für den
praktischen Einsatz ungeeignet macht. Dabei wird der größte Teil der Zeit für Zugriffe auf den Hashtable aufgewendet (siehe Abschnitt \ref{ksearch}). Dies ist in sofern nicht
überraschend, dass wir uns aufgrund des massiven Speicherplatzbedarfs unserer ursprünglichen Implementierung für einen geringeren
Speicherplatzverbrauch entschieden, dafür jedoch bewusst Geschwindigkeitseinbußen in Kauf genommen haben.

Der implementierte \emph{SANS}-Algorithmus benötigt lange für die Erstellung der Suffix-Arrays, zeigt jedoch bei zunehmender Fensterbreite \textit h 
eine weniger stark ansteigende Laufzeit als der Referenzalgorithmus. Mit Ausnahme des implementierten \emph{KSEARCH} zeigen alle Algorithmen eine 
signifikant schnellere Laufzeit als BLAST und SSEARCH mit einer Laufzeitersparnis von teilweise mehr als 50 \%.
  \begin{table}[h]
    \caption{Laufzeitenvergleich; KH entspricht den Algorithmen aus \cite{Holm}, MMF kennzeichnet die selbst implementierten Algorithmen; Berechnungen durchgeführt auf: Intel i5, 4*3.4GHz, 32gb RAM, XUbuntu}
    \begin{center}
    \begin{tabular}{lrrr}
    \hline
    Algorithmus & Indexing & Suche & Gesamt\\
    \hline
    BLAST & 0m12s & 27m23s & 27m35s\\
    SSEARCH & - & 60m0s & 60m0s\\
    KSEARCH KH k6 & 1m29s & 0m14s & 1m43s\\
    KSEARCH MMF k6 & -  & 511m53s & 511m53s\\
    SANS KH h16 & 1m29s &  0m15s & 1m44s\\
    SANS MMF  h16 & 14m19s & 0m36s & 14m55s\\
    SANS KH h1000 & 1m29s & 7m7s & 8m36s\\
    SANS MMF  h1000 & 14m19s & 1m56s & 16m15s\\
    \hline
    \end{tabular}
    \label{runtimes}
    \end{center}
  \end{table}

\subsection{Scores}

Das wichtigste Kriterium bei der Beurteilung der Benutzbarkeit unserer Implementierungen ist die Qualität der Ergebnisse. Um hier Aussagen treffen
zu können, haben wir einerseits verglichen, wie viele der mit SSEARCH ermittelten Treffer auch mit den jeweiligen Algorithmen wiedergefunden werden konnten; 
andererseits wurde mittels AUC-Berechnungen die Effizienz der Algorithmen getestet.

\subsubsection{Vergleich mit SSEARCH}

%SANS
Um einen ersten Eindruck der Güte unseres \emph{SANS} Algorithmen zu erhalten, verglichen wir zunächst die besten 1000 Scores unserer Implementierung
mit der aus \cite{Holm}. Dabei waren $919$ gleiche Sequenzpaare unter diesen am besten bewerteten $1000$. Es sind also $81$ sich unterscheidende
Sequenzpaare dabei. Generell konnten wir beobachten, das die gleichen Testsets nicht zu exakt gleichen Scores bei beiden Algorithmen führten. Die
Abweichungen, die wir hier feststellen konnten, waren jedoch im Vergleich zum Score gering.

Die Scores unterscheiden sich überhaupt, da einige Implementierungsdetails in \cite{Holm} nicht erschöpfend und nachvollziehbar beschrieben wurden,
und so an einigen Stellen eigene Entscheidungen getroffen werden mussten. Hier ist beispielsweise die Sortierung der Aminosäuren von Proteinsequenzen
zu nennen. Hier ist sowohl eine rein lexikographische als auch eine nach biologischen Kriterien vorgenommene Sortierung denkbar. Aus unterschiedlichen
Sortierungen resultiert dabei, dass Suffixe des querys an anderen stellen der Datenbank einsortiert werden. Somit ändern sich natürlich auch die 
entsprechenden Scores.

Weiterhin wurde, basierend auf \cite{Holm}, die Sensitivität beider Implementierungen verglichen. Dazu wurde zunächst SSEARCH auf den Rattus Query
Sequenzen mit der Swissprot Datenbank durchgeführt. Diese Referenzergebnisse wurden anschließend in den besten $1000$ Treffern der beiden SANS-Algorithmen
für jedes Query-Datenbank Paare gesucht. Wie in \cite{Holm} erfolgte eine Aufteilung in Identitätsklassen. Abbildung \ref{barplot} zeigt, dass beide
Implementierungen eine vergleichbare Sensitivität erreichen.

\begin{figure}
  \begin{center}
    \includegraphics[width=0.6\textwidth]{img/barplot_beide.png}
    \caption{Sensitivität des selbst implementierten SANS-Algorithmus und dem aus \cite{Holm}. Als Referenz dienen SSearch-Alignments.
    Die Treffer sind nach Identität der Sequenzpaare eingeteilt.}
    \label{barplot}
  \end{center}
\end{figure}

\subsubsection{Area under the curve (AUC)}


Um die Effizienz eines Algorithmus zu ermitteln, wird ein precision-recall-Diagramm erstellt, und daraus die AUC berechnet. Die AUC ist die Fläche unter der Kurve, also das Integral. Das precision-recall-Diagramm erhält man, indem Präzision (precision) $p$ gegen die Wiederfindungsrate (recall) $r$ geplottet wird. Beide Werte werden errechnet, indem als erstes eine Liste \textit l mit einem Referenz-Algorithmus erstellt, und anschließend anhand dieser Liste \textit p und \textit r des zu testenden Algorithmus ermittelt werden.\\In Liste \textit l wurden alle SSEARCH-Treffer aus dem Query-Datensatz, welche einen e-Value niedriger als eins aufwiesen, eingetragen. Anschließend wurde mit den bestem 1000 Treffern \textit t jedes Query-Proteins Präzision (precision) $p = \frac{t_0 ... t_i \in l}{t_0 ... t_i}$ und Wiederfindungsrate (recall) $r = \frac {t_0 ... t_i \in l}{t \in l}$ berechnet und das precision-recall-Diagramm erstellt. Dabei gibt die Wiederfindungsrate \textit r an, wie viele der richtigen Treffer von \textit t zum Zeitpunkt \textit i bereits gefunden wurden. Die Präzision \textit p gibt die Anzahl der richtigen Treffer, im Verhältnis zu allen zum Zeitpunkt \textit i bereits betrachteten Treffern, an. Da \textit p und \textit r Werte zwischen 0 und 1 annehmen, hat die AUC im besten Fall den Wert 1. Im Allgemeinen nimmt mit zunehmender Wiederfindungsrate die Präzision ab, da bei einem kleinen AUC-Wert viele false positives auftreten, bis die maximale Anzahl aller richtigen Treffer ermittelt ist. Je kleiner also der Wert für AUC, desto ineffizienter ist der Algorithmus.\\Zu beachten ist, dass beim Plotten von \textit p gegen \textit r nur ein Punktdiagramm entsteht. Die darin enthaltenen Punkte können nicht interpoliert werden, wodurch die Kurve für die AUC nur näherungsweise bestimmt werden kann.\\Die Ergebnisse sind in Tab. 2 veranschaulicht. Es wurde für jedes query-Protein eine AUC erstellt, und daraus der Durchschnitt berechnet. Die Werte aus \cite{Holm} konnten bestätigt werden.

  \begin{table}[h]
    \centering
    \caption{Werte für die AUC; je höher der Wert, je effizienter arbeitet der Algorithmus}
    \begin{tabular}{cc}
      \hline
      SANS KH & SANS MMF\\
      h = 1000 & h = 1000 \\
      \hline
      0,9677 & 0,9291 \\
      \hline
    \end{tabular}
  \end{table}


\addcontentsline{toc}{section}{Literatur}
\bibliography{quellen}{}
\bibliographystyle{unsrt}
\pagebreak
\section{Anhang}
\subsection*{bla\_seqscore.h}
\lstinputlisting{bla_seqscore.h}

\end{document}
